    \begin{center}
      \vspace{.1cm}
      \section*{Methodology}
      \end{center}
      \vspace{-.05cm}
      Our study aimed to predict the ROP in a drilling dataset using machine learning algorithms. Our key steps includes 
    
    \begin{itemize}
    \item Data Collection: Gathered a drilling dataset that included various drilling parameters, such as weight on bit, surface torque, rotary speed, etc. We also included the target variable, ROP.
      	\item Data Preprocessing: Cleaned the dataset by handling missing values, data imputation, data normalization, feature selection or engineering, and splitting the data into training and testing sets.

\item Algorithm Selection: Selected four machine learning algorithms, including Support Vector Machine (SVM), Random Forest, Linear Regression, Artificial Neural Network (ANN), and k-Nearest Neighbor (KNN).

\item Model Training: Trained each algorithm on the training data and optimized the hyperparameters using grid search or randomized search.

\item Model Evaluation: Evaluated the performance of each model on the testing data using various evaluation metrics such as MSE, MAE, and R2 score.

\item Model Deployment: Selected the best-performing model(s) for real-world applications.

\item Model Maintenance: Monitor the model's performance regularly and update the model as necessary to improve its accuracy or to adapt to new data.

      \end{itemize}
